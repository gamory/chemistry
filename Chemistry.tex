\documentclass{book}
\usepackage{makeidx}
\usepackage{hyperref}
\usepackage{graphicx}
\hypersetup{colorlinks=true}
\makeindex
\author{Richard "teslafreak" Burris}
\title{Chemistry}
\pagestyle{headings}
\begin{document}
\maketitle
\newpage
\tableofcontents
\newpage
\chapter{Preface}
This is an attempt at gathering information from many disparate sources, and gathering them into one handy source.  I can try to souce things where possible, but it may not always be.  I have always been interested in Chemistry, and indeed many facets of science.  However, I was not well enough invested learning these things when I was in public school.  Luckily in the modern age of the internet, there are many fantastic resources, and you're never to old to learn new things.\\
\\
That said, I do want to be clear that I am an Information Technology worker by trade.  Not a chemist.  Not a doctor.  Not a physicist.\\
\\
I will make every reasonable attempt to ensure that all data is accurate, and that any risks I am aware of are stated, but these fields have many inherent risks, and death is a pretty common one.  Please see the warning chapter, the general risks chapter, and read any warnings on individual subjects.  I can't be held responsible for anything bad that happens, even if you follow these instructions to the letter.\\
\\
The content of this book is intended to be as wide as I can make it, so it probably won't be a fun read through as a novel (not to mention, I am a terrible writer).  It's primary usage should be as a reference.\\
\\
Where a specific company is named, rest assured that as of this writing, it is a reference only.  I am not sponsored by anyone, and have paid my own money for all equipment herein.\\
\\
Lastly, let me just apologize in advance for my poor writing again.  Poor sentence structure, grammer, and likely even spelling will probably pepper this document.  Corrections are accepted and welcomed in any regard you may have.  Even excepting the aforementioned issues, i've always felt I was a poor writer, producing mostly quite dull content.  I apologize if things are overly dry.  
\chapter{Warning}
While science can be a great deal of fun, the ways in which you can run into trouble are \textbf{numerous}.\\
\section{Legal}
First and probably worth consideration before you really start in on this: is it legal?\\
\\
Some countries/states/locations don't really allow you to do anything, and some don't really care as long as you don't directly break laws.  The most common thing people worry about seems to be what they could legally make.  That is a fair concern, and many places won't really have much concern as long as you don't make drugs, explosives, or meddle in radioactive substances.  Some places will even work with you on some of that (notably explosives) as long as your local law enforcement is aware.  A number of years back, I received permission to do small (less than a gram) experiments with black powder as long as it was out in the hills away from any homes and I called to let the police know when I was working on it so they would know what was going on if anyone called it in.  Your milage may vary.  By a lot.  You could have some very understanding local law enforcement that will work with you as long as you are upfront with them and can explain your safety measures.  You could get law enforcement that not only says no, but now wishes to keep an eye on you.\\
\\
Unfortunately, this is a situation that is really only likely to get worse because there is a strange perception that science in general (and things like chemistry and biology in particular) should strictly be the domain of institutions and businesses.  If you want to learn this stuff as an individual, it's not uncommon for people to assume the only possible reason is to make illicit substances.  This has made things even more convoluted by causing to more very important considerations:  Can I even buy (a chemical)?  Can I even buy (a piece of equipment)?\\
\\
Some chemicals that have a massive number of very legitamate uses are restricted or banned (elemental Iodine comes to mind) purely under the concept that you \textit{could} use them to make illicit substances.  This is, I will say, an unabashedly stupid policy for much the same reason that you could use nearly any weighted solid object to illegally bludgeon someone to death.  Things very much should be consider on the pure and strict basis of what they are used for, not what they could be used for.  This is not the reality of the situation though, and as a result, you must be aware of your local laws.  I can only speak for the US, as I have only ever lived here.  The biggest consideration in this regard will be the DEA controlled chemicals list.  You are encouraged to keep familiar with it.  For most chemicals, it doesn't outright ban you from having them, but merely sets limits on how much of them you can have.  Some states may have additional restrictions, so be aware of that.  Probably the biggest problem with this list is that although it shouldn't matter to someone that wants (let's say) a couple grams of a subtance on it, it does cause havoc for the chemistry store you may want to purchase from.  When they need to order some of these chemicals in bulk, they are required to fill out and maintain paperwork detailing what they are ordering, how much, what for, etc.  For many sources, they've decided they just don't want to bother.  This means that although it would be ok for you to buy it, you may have a hard time sourcing it.  The internet is going to be your friend here.\\
\\
Additionally, some equipment is actually forbidden.  I personally believe this is absurd as well, but I have heard there are places that don't allow jointed glassware.  Yes seriously.  Apparently the mere act of putting a nice mating joint on a piece of glass makes it malicious.  I am not sure what other equipment would be, but I can reasonably guess that this isn't the only example.\\
\\
So, if you have found that it's legal to have some chemicals, and some equipment, and you aren't planning to break any laws with it; are you good now?  You probably won't be shocked to hear that you may still not be.  Some (and really many) places have policies that can cause otherwise legal things to become additional charges or evidense should you become legally entangled.  In anything.  For instance, where I live (and at this time) in Utah, it is legal to own lock picks.  Many places consider lock picks to be burglery tools, and ban them outright, but not Utah.  However, in the event that you were say kicked out of a property and the police found that you had some; even if you didn't use them; they are burglery devices now.  I have heard the same thing can happen where they can magically decide you have chemical pre-cursors or glassware that could be used for crimes, and now they are evidense that you surely must be involved in such crimes.  The laws around using such "evidense" are murky and if in doubt, I would strongly suggest speaking to a lawyer.  As mentioned in the preface, I am not one.\\
\\
Legal considerations aside, the biggest suggestion I would make here is to avoid even the appearance of criminality.  This may sound cliched, but really, people will be far more likely to trust you if there are no reasons not to.  If you have a prior drug charge, and are caught with certain materials; you may well be hosed.  It doesn't matter in that case if you are clean now, people will often take the dimmest view of a situation that they can.  You should try not to have any real cause for them to do so.
\section{Safety}
I don't want to get too specific here, as I think the better place will be general risks and individual safety instructions.  What I do want to point out is that the equipment, substances, and methods herein constitute everything from a completely safe situation with no perceptible risk, all the way up to situations that may have several decent ways to kill you and/or those in the area for a single experiment.  It should really go without stating that you \textbf{must exercise proper safety procedures}.  This can vary widely, and I will give you the best guidance I can.  It also pays to read up on your own.\\
\\
The most general advise would be:\\
\begin{itemize}
\item Always ensure good ventilation.  This isn't just for fumes or gasses, it's also not uncommon for dusts to get lofted in to the air.  Better safe than sorry.  Have you ever heard stories of toxic homes because someone used to make meth there?  Same idea.  Chemicals can accumulate and become dangerous over time, and it may not even be obvious.
\item Always ensure safe storage.  Not only so no one gets into things they shouldn't, but also to ensure a leak doesn't involve a bad interaction.  Some chemicals also don't store well at certain temperatures.  Some store just fine, but will decay faster.  Read up on proper storage procedures.
\item Always ensure safe disposal.  Many chemicals are not safe to put in the garbage or down the drain.  The risks can range from just putting something there that shouldn't be, to severely poisoning anything downstream (or groundwater).  It also may be illegal, but really that should be a secondary concern to just being a good person.
\item Always wear proper attire and PPE (personal protective equipment).  At a bare minimum, this is safety glasses.  In most circumstances it will also include a lab coat and gloves, maybe a respirator.  Some people think the lab coat is just to look cool (and boy does it!) but it's actually there to protect you from spills.  Get one, and wear it.
\item Always follow all safety procedures.  You never know when something will go wrong, and you will be happy that you mitigated it as much as possible.
\item Never skip safety "this time".  It's very easy to be complacent.  It's very easy to decide you just don't want to deal with it "this time" because it's "not a concern".  This mentality \textbf{kills}.  If you don't have time to do it right, don't do it.
\item Never leave a project un-supervised.  If it's boiling now, it could boil over.  If it's un-covered, it could get spilled.  Act appropriately.
\item Be cautious of anyone else involved.  This is everything from making sure people in the area or who may help directly are appropriately safe and knowledgeable; to making sure that if you provide a chemical or equipment to someone, you know them and trust they will use it appropriately.  If they don't, you can be legally liable in some circumstances.
\item Let someone know what you are doing.  In the event that you become incapacitated, this could save your life.  I am not saying it's likely (it's quite unlikely if you are being safe) but again, hedging your bets towards safety is good policy.
\item Lastly, have some common sense.  This is not a hobby to practice while drunk/high/otherwise in an altered state.  It's not safe to do risky things when tired either.  I won't tell you how to live your life, but being "clean" in this regard will also mitigate your chances of getting involved in a call to law enforcement, which again, is prudent.
\end{itemize}

Many chemicals will state hazards they could present such as poisonous, flammable, explosive, corrosive, etc.  It's important to understand how these properties relate to their concentrations.  For instance, Hydrogen Peroxide in it's common store bought form at 3\% is something many people would not hesitate to dump all over an open wound (indeed, that is what it's commonly sold for).  If you bring that up to 30\%, it will cause chemical burns.  Even more concentrated and those burns can be quite severe.  Some acids in dilute concentrations present very minor threat to skin when washed off quickly, but will put you in a hospital in concentrated forms (or kill you outright).  Some chemicals have no safe limit whatsoever.  If you get so much as a drop of methylmercury on your skin, you're not long for this world (mercury posioning will set in, and is fairly universally fatal).  Worse still, methylmercury goes through gloves, so that won't even protect you.  Thankfully methylmercury is something you are unlikely to ever come in contact with, and isn't really a useful chemical to have around either.  There are plenty of chemicals that are nasty though, and which you have a good chance of coming across.  Some present interesting safety considerations by the way they interact with the environment.  You probably know that you shouldn't breath chloroform, but did you know that in regular atmosphere it can turn into phosgene?  Phosgene is so utterly nasty that it was used as a chemical weapon.  One is kinda bad for you, the other is extremely toxic.\\
\\
Possibly my favorite example is Nitrogen Trichloride, a great video is \href{https://www.youtube.com/watch?v=mV_daaldE_I}{here} (this is a video by Tom who runs the YouTube channels \href{https://www.youtube.com/c/ExplosionsFire2}{Explosions\&Fire} and \href{https://www.youtube.com/c/ExtractionsIre}{Extractions\&Ire}, I can't recommend his content highly enough, it's a decent contender for my favorite Youtube channel).  For those that can't or don't wish to watch it, the premise is that it's very easy to have it form by accident.  When you are in a pool, and something is irritating your eyes, it's probably this (people often suspect it to be chlorine itself, but that is not the case).  In this terrifically low concentration, it's merely unpleasant to your eyes.  When concentrated, not only is it very poisonous, it's an extrordinarily unstable explosive!  To my mind, it's these very sorts of things that add to the fascination of chemistry, but it also serves to remind you that risks are inherent and not always obvious.\\
\section{Reference materials and resources}
\subsection{YouTube channels}
\href{https://www.youtube.com/c/ChemicalForce}{ChemicalForce}\\
\href{https://www.youtube.com/user/DougsLab}{Doug's Lab}\\
\href{https://www.youtube.com/c/ExplosionsFire2}{Explosions\&Fire}\\
\href{https://www.youtube.com/c/ExtractionsIre}{Extractions\&Ire}\\
\href{https://www.youtube.com/c/NileRed2}{NileBlue}\\
\href{https://www.youtube.com/c/NileRed}{NileRed}\\
\href{https://www.youtube.com/user/periodicvideose}{Periodic Videos}\\
\href{https://www.youtube.com/c/Thoisoi2}{Thoisoi2}\\
\subsection{Websites}
\href{https://iupac.org/}{International Union of Pure and Applied Chemistry}\\
\href{https://www.rsc.org/}{Royal Society of Chemistry} - A fantasic resource for information on elements\\
\chapter{General Safety}
\chapter{Equipment}
\chapter{The Elements}
This chapter will be by atomic number rather than alphabetic.
\section{1 - Hydrogen}
\label{sec:elem-hydrogen}
\section{2 - Helium}
\label{sec:elem-helium}
\section{3 - Lithium}
\label{sec:elem-lithium}
\chapter{Chemicals}
\chapter{Compositional reference}
\hyperref[sec:elem-hydrogen]{H}\\
\hyperref[sec:elem-helium]{He}\\
\hyperref[sec:elem-lithium]Li}\\
\chapter{CAS reference}
CAS references will vary in structure as to how many digits are seperated by commas.  I believe they are always a group of three numbers (so two hyphens), but in the interest of figuring out a sort order, I have chosen to interpret the number as though it was a solid number with no seperations.  I'm not sure if this is the preferred method, but it's the one you will see here.\\
\\
\hyperref[sec:elem-hydrogen]{133-74-0 - Hydrogen}\\
\hyperref[sec:elem-lithium]{7439-93-2 - Lithium}\\
\hyperref[sec:elem-helium]{7440-59-7 - Helium}\\
\chapter{IUPAC reference}
\chapter{Glossary}
\end{document}